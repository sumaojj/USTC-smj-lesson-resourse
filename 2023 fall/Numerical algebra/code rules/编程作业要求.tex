\documentclass{article}
\usepackage[UTF8]{ctex}
\usepackage{float,indentfirst,verbatim,fancyhdr,graphicx,listings,longtable,amsmath, amsfonts,amssymb}

\textheight 23.5cm \textwidth 15.8cm
%\leftskip -1cm
\topmargin -1.5cm \oddsidemargin 0.3cm \evensidemargin -0.3cm

%\pagestyle{fancy} \lhead{FDM Homework Template} \chead{}
%\rhead{\bfseries}
%
%\lfoot{} \cfoot{} \rfoot{\thepage}
%\renewcommand{\headrulewidth}{0.4pt}
%\renewcommand{\footrulewidth}{0.4pt}
%
\title{编程作业要求}
\author{游瀚哲}

\begin{document}
\maketitle

\section*{一、实验程序}

编程语言使用 c++。但实际上,本课程的算法只涉及矩阵的运算,只需使用vector类即可满足需求。
在不调用vector类其他函数的情况下,vector类和C语言中的数组有很多相似之处。

$vector	\left \langle double \right \rangle b(N); $定义一个长为N的零向量

$vector \left \langle vector\left \langle double \right \rangle \right \rangle  A(M, vector\left \langle double \right \rangle(N))$; 
定义一个M*N的零矩阵

b[0]; 访问b第一个元素
 
A[0][0] ;访问A第一行第一列的元素

$void \; forward\_ subs(vector \left \langle vector\left \langle double \right \rangle \right \rangle \& L, vector	\left \langle double \right \rangle \& b)$ 
矩阵和向量在函数中的形参格式

现在你已经学会这门课所需的c++知识了,快试试用c++写程序作业吧(雾),
当然,使用其他数据结构用于表示矩阵也是可以的。项目创建可以参考提供的代码框架。

IDE尽量使用visual studio。相比于dev-c++,其默认的调试界面更有利于发现错误,
养成良好的编程习惯(eg:合适的缩进,合理的变量名,合适的项目结构与排版,适当的注释函数使用方式)对debug很有好处。

visual studio可以直接在官网下载,社区版已经足够满足本课程的要求,且无需进行环境变量的配置。 



\section*{二、实验报告}

这门课应该算是数院同学接触的第一门要写实验报告的数学课,
学会写一手漂亮的报告对以后的学习研究有重要的意义,
故在此给出一些要求和帮助。

在群文件中提供了word和latex的两种模版,
latex相关软件可以在学校官网正版软件上下载,
模版可应对报告大部分需求,细节问题直接搜索即可。
但不管采用何种模版,一定要将提交报告转化为pdf格式。

实验报告应该至少 包括以下几个部分:

问题描述:简单介绍你要解决的问题。

程序介绍:提出你所使用的算法,如果有与算法相关的参数请指出,并简要介绍程序的使用方式。

实验结果:展示实验所产生的结果,例如程序运行产生的结果截图,(注:不建议将向量打印成列向量)。
如果愿意的话,可以将程序运行所得的数据列表展示。

结果分析:分析得到的实验结果,例如对比多个算法的精度和耗时,或分析实验结果的数值性态等。
还可以从理论和算法的角度分析出现这些现象的原因是什么。

在批改实验时,对于报告格式不会太过严苛,\textbf{但清晰的展示实验结果是必须的}。
尽管如此,还是鼓励提交更详细更规范的实验报告。

\section*{三、提交要求}

上交作业时,需要删除项目文件中的.vs文件(以防止内存过大无法发送),再将整个项目和sln文件放入压缩包,
保证\textbf{程序是完整且可运行的},将报告pdf也放入压缩包,并将压缩包命名为学号+姓名。
将压缩包发送到邮箱 ustcszds2020@163.com,将主题也命名为学号+姓名。
如果一切顺利,会在提交作业后两周之内收到打分和反馈。

\textbf{可以参考讨论但是严禁 copy 前几届或是同学的代码,应老师要求,一经发现本次作业 0 分处理。}

\end{document}